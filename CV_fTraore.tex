%!TEX TS-program = xelatex
%!TEX encoding = UTF-8 Unicode
% Awesome CV LaTeX Template for CV/Resume
%
% This template has been downloaded from:
% https://github.com/posquit0/Awesome-CV
%
% Author:
% Claud D. Park <posquit0.bj@gmail.com>
% http://www.posquit0.com
%
%
% Adapted to be an Rmarkdown template by Mitchell O'Hara-Wild
% 23 November 2018
%
% Template license:
% CC BY-SA 4.0 (https://creativecommons.org/licenses/by-sa/4.0/)
%
%-------------------------------------------------------------------------------
% CONFIGURATIONS
%-------------------------------------------------------------------------------
% A4 paper size by default, use 'letterpaper' for US letter
\documentclass[11pt, a4paper]{awesome-cv}

% Configure page margins with geometry
\geometry{left=1.4cm, top=.8cm, right=1.4cm, bottom=1.8cm, footskip=.5cm}

% Specify the location of the included fonts
\fontdir[fonts/]

% Color for highlights
% Awesome Colors: awesome-emerald, awesome-skyblue, awesome-red, awesome-pink, awesome-orange
%                 awesome-nephritis, awesome-concrete, awesome-darknight

\definecolor{awesome}{HTML}{414141}

% Colors for text
% Uncomment if you would like to specify your own color
% \definecolor{darktext}{HTML}{414141}
% \definecolor{text}{HTML}{333333}
% \definecolor{graytext}{HTML}{5D5D5D}
% \definecolor{lighttext}{HTML}{999999}

% Set false if you don't want to highlight section with awesome color
\setbool{acvSectionColorHighlight}{true}

% If you would like to change the social information separator from a pipe (|) to something else
\renewcommand{\acvHeaderSocialSep}{\quad\textbar\quad}

\def\endfirstpage{\newpage}

%-------------------------------------------------------------------------------
%	PERSONAL INFORMATION
%	Comment any of the lines below if they are not required
%-------------------------------------------------------------------------------
% Available options: circle|rectangle,edge/noedge,left/right

\photo{pic.jpeg}
\name{Fatoumata-Adama TRAORE}{}

\position{Ingénieur en Bioinformatique - Disponibilité immédiate}
\address{152 Avenue Gabriel Peri, Bezons}

\mobile{07 82 31 61 65}
\email{\href{mailto:traorefadamaa@gmail.com}{\nolinkurl{traorefadamaa@gmail.com}}}
\linkedin{FATraore}

% \gitlab{gitlab-id}
% \stackoverflow{SO-id}{SO-name}
% \skype{skype-id}
% \reddit{reddit-id}


\usepackage{booktabs}

% Templates for detailed entries
% Arguments: what when with where why
\usepackage{etoolbox}
\def\detaileditem#1#2#3#4#5{%
\cventry{#1}{#3}{#4}{#2}{\ifx#5\empty\else{\begin{cvitems}#5\end{cvitems}}\fi}\ifx#5\empty{\vspace{-4.0mm}}\else\fi}
\def\detailedsection#1{\begin{cventries}#1\end{cventries}}

% Templates for brief entries
% Arguments: what when with
\def\briefitem#1#2#3{\cvhonor{}{#1}{#3}{#2}}
\def\briefsection#1{\begin{cvhonors}#1\end{cvhonors}}

\providecommand{\tightlist}{%
	\setlength{\itemsep}{0pt}\setlength{\parskip}{0pt}}

%------------------------------------------------------------------------------



\begin{document}

% Print the header with above personal informations
% Give optional argument to change alignment(C: center, L: left, R: right)
\makecvheader

% Print the footer with 3 arguments(<left>, <center>, <right>)
% Leave any of these blank if they are not needed
% 2019-02-14 Chris Umphlett - add flexibility to the document name in footer, rather than have it be static Curriculum Vitae
\makecvfooter
  {juin 2020}
    {Fatoumata-Adama TRAORE~~~·~~~Curriculum Vitae}
  {\thepage}


%-------------------------------------------------------------------------------
%	CV/RESUME CONTENT
%	Each section is imported separately, open each file in turn to modify content
%------------------------------------------------------------------------------



\hypertarget{expuxe9riences-professionnelles}{%
\section{Expériences professionnelles}\label{expuxe9riences-professionnelles}}

\detailedsection{\detaileditem{Ingénieur bioinformaticienne}{Octobre 2018 - Janvier 2020 (16 mois)}{ INRA - Unité MetagenoPoliS }{}{\item{Préparation et analyse de données métagénomiques}\item{Comparaison et annotation de séquences}\item{Construction de catalogue de gènes du microbiote intestinal}\item{Reconstruction d’espèces phagiques}}\detaileditem{Ingénieur bioinformaticienne}{Janvier 2018 - Juillet 2018 (6 mois)}{ INRA - Unité MetagenoPoliS }{}{\item{Identification de gènes de phages du microbiote intestinal }\item{Test d’outils et paramètres de traitement de données métagénomiques}\item{Construction d’un catalogue de gène}}\detaileditem{Bioinformaticienne}{Mars 2017 - Mai 2017 (3 mois)}{ Institut Jacques Monod }{}{\item{Recolte et homogéniéisation de données}\item{Mise en place de modèles mathématiques pour definir différentes populatins dans les données}\item{Normalisation de l'expression des gènes}\item{Regroupement des cancers selon leur ressemblance}}\detaileditem{Bioinformaticienne}{Juin 2016 - Juillet 2016 (1 mois)}{ Laboratoire de Biochimie - ESPCI }{}{\item{Mise en place d’un programme de traitement de données de séquençage à haut de,it}}\detaileditem{Bioinformaticienne}{Juin 2015 - Juillet 2015 (1 mois)}{ Centre de Recherche sur les Macromolécules Végétales  }{}{\item{Actualisaion et annotation d’une base de données de structures tridimensionnelles de glucides}}}

\hypertarget{formation}{%
\section{Formation}\label{formation}}

\detailedsection{\detaileditem{Universite Paris Diderot Paris 7}{2015 - 2018}{}{}{\item{ Master 1 et 2 Bioinformatique}\item{Licence Biologie Informatique Bioinformatique}}\detaileditem{Universite Joseph Fourier Grenoble 1}{2013 - 2015}{}{}{\item{Licence 1 et 2 Biologie et informatique}}}

\hypertarget{compuxe9tences---informatiques---langues}{%
\section{Compétences - Informatiques - Langues}\label{compuxe9tences---informatiques---langues}}

\detailedsection{\detaileditem{Compétences}{}{}{}{\item{ Bioinformatique}\item{Programmation}\item{Analyses de données NGS}\item{Statistiques}\item{Base de données}}\detaileditem{Informatiques}{}{}{}{\item{Pyhton - R - Linux - Perl - PostgreSQL - HTML/CSS - Java}}\detaileditem{Langues}{}{}{}{\item{Français : Courant}\item{Anglais :  Intermediaire}}}

\end{document}
